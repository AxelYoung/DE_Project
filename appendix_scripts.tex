\subsection{Deterministic Two-Stage Model Script (script.py)}
\begin{lstlisting}[language=Python, caption={Python script for fitting a two-stage piecewise deterministic SEIAR model, corresponding to the pre- and post-lockdown phases.}]
import pandas as pd
import numpy as np
from scipy.integrate import odeint
from scipy.optimize import curve_fit
import matplotlib.pyplot as plt

# --- Step 1: Import libraries and load data ---

# Load the processed German epidemic data
try:
    df = pd.read_csv('germany_covid_processed.csv')
    # Core fitting data: estimated total current infectious individuals
    i_data = df['infectious_pool'].values
    t_data = np.arange(len(i_data))
    # Get total population N
    N = df['population'].iloc[0]
except FileNotFoundError:
    print("Error: Data file 'germany_covid_processed.csv' not found.")
    print("Please ensure you have run data_processing.py to generate this file.")
    exit()

# --- Global parameters and initial conditions ---
# Fixed parameters (from literature in Part 2 of our report)
F = 0.17       # Asymptomatic proportion
SIGMA = 1/4.5    # Latent period transition rate (1/day)
GAMMA_I = 1/19.5 # Symptomatic recovery rate (1/day)
GAMMA_A = 1/17.0   # Asymptomatic recovery rate (1/day)

# Initial conditions
I0 = 0      # Assume no diagnosed symptomatic individuals at t=0
A0 = 0      # Assume no asymptomatic individuals at t=0
E0 = 1      # Key: Assume 1 exposed individual enters the population at t=0, as the "seed" of the epidemic
R0 = 0
S0 = N - I0 - E0 - A0 - R0
Y0 = (S0, E0, I0, A0, R0) # Initial state tuple

# --- Step 2: Define the SEIAR model function ---

# Define the differential equations for the SEIAR model
# We take all parameters as inputs to increase the function's versatility
def seiar_model(y, t, N, beta_I, f_param, sigma_param, gamma_i_param, gamma_a_param):
    """
    ODE system for the SEIAR model.
    beta_I is now a parameter, not a global variable.
    """
    S, E, I, A, R = y
    
    # Calculate beta_A based on the relationship from literature
    beta_A = 0.58 * beta_I
    
    # Differential equations
    dSdt = -beta_I * S * I / N - beta_A * S * A / N
    dEdt = beta_I * S * I / N + beta_A * S * A / N - sigma_param * E
    dIdt = (1 - f_param) * sigma_param * E - gamma_i_param * I
    dAdt = f_param * sigma_param * E - gamma_a_param * A
    dRdt = gamma_i_param * I + gamma_a_param * A
    
    return dSdt, dEdt, dIdt, dAdt, dRdt

# --- Step 3: Create a new, more powerful "solver" function ---

def fit_odeint_piecewise(t, beta_pre, beta_post, rho, E0_fit):
    """
    An advanced solver for curve_fit.
    It uses a piecewise-changing beta and simultaneously fits the observation proportion rho and initial exposed count E0.
    """
    t_switch = 22  # Approximate date of Germany's national lockdown (2020-03-22), corresponding to day 22 in the data

    # Update initial conditions with the fitted parameters
    # E0 is to be fitted; I0, A0, R0 start from 0
    y0_fit = (N - E0_fit, E0_fit, 0, 0, 0)

    # Split the time array into "pre-lockdown" and "post-lockdown"
    t_pre = t[t < t_switch]
    t_post = t[t >= t_switch]

    # Create an empty numpy array to store the combined infected population
    infected_combined = np.array([])
    
    # Solve for the first phase (pre-lockdown)
    if len(t_pre) > 0:
        # Solve using the pre-lockdown beta
        res_pre = odeint(seiar_model, y0_fit, t_pre, args=(N, beta_pre, F, SIGMA, GAMMA_I, GAMMA_A))
        # Extract the I+A part
        infected_pre = res_pre[:, 2] + res_pre[:, 3]
        infected_combined = np.concatenate((infected_combined, infected_pre))
        # The initial condition for the second phase is the end point of the first phase
        y0_post = res_pre[-1]
    else:
        # If the fitting window is entirely after the lockdown, the initial condition for the second phase is our specified y0_fit
        y0_post = y0_fit

    # Solve for the second phase (post-lockdown)
    if len(t_post) > 0:
        # Solve using the post-lockdown beta
        res_post = odeint(seiar_model, y0_post, t_post, args=(N, beta_post, F, SIGMA, GAMMA_I, GAMMA_A))
        # Extract the I+A part
        infected_post = res_post[:, 2] + res_post[:, 3]
        infected_combined = np.concatenate((infected_combined, infected_post))

    # Return the model prediction scaled by the observation proportion rho
    return rho * infected_combined

# --- Step 4: Perform the fit and visualize ---

# Define the fitting window
fit_duration = 180
t_fit = t_data[:fit_duration]
i_fit = i_data[:fit_duration]

# Provide initial guesses and bounds for the 4 parameters to be fitted
# Parameter order: [beta_pre, beta_post, rho, E0_fit]
initial_guesses = [0.6, 0.1, 0.2, 100]
bounds = (
    [0.1, 0.01, 0.01, 1],      # Lower bounds
    [2.0, 1.0, 1.0, 50000]     # Upper bounds
)

# Perform the fit
try:
    popt, pcov = curve_fit(
        fit_odeint_piecewise,
        t_fit, 
        i_fit, 
        p0=initial_guesses, 
        bounds=bounds,
        maxfev=3000  # Increase max iterations to help convergence
    )
    
    # Extract optimal parameters
    beta_pre_opt, beta_post_opt, rho_opt, E0_fit_opt = popt
    
    print("Fit successful!")
    print(f"Pre-lockdown transmission rate (beta_pre): {beta_pre_opt:.4f}")
    print(f"Post-lockdown transmission rate (beta_post): {beta_post_opt:.4f}")
    print(f"Case observation proportion (rho): {rho_opt:.4f}")
    print(f"Initial exposed population (E0_fit): {E0_fit_opt:.2f}")

    # --- Visualization ---
    
    # Rerun the solver with optimal parameters to generate the final fitted curve
    fitted_curve = fit_odeint_piecewise(t_fit, *popt)
    
    # Create the plot
    plt.figure(figsize=(14, 9))
    # Plot all real data points (entire time series)
    plt.plot(t_data, i_data, 'o', label='Real Data (Estimated Total Infectious)', markersize=4, alpha=0.5)
    # Plot the fitted curve within the fitting window
    plt.plot(t_fit, fitted_curve, 'r-', label='Fitted SEIAR Model', linewidth=3)
    
    # Mark the lockdown date with a vertical line
    t_switch = 22
    plt.axvline(x=t_switch, color='k', linestyle='--', label=f'Lockdown Start (Day {t_switch})')
    
    # Annotate the plot with the optimal parameter values
    param_text = (
        f'Fitted Parameters:\n'
        f'$\\beta_{{pre}}$ = {beta_pre_opt:.3f}\n'
        f'$\\beta_{{post}}$ = {beta_post_opt:.3f}\n'
        f'$\\rho$ (rho) = {rho_opt:.3f}\n'
        f'$E_0$ = {E0_fit_opt:.0f}'
    )
    plt.text(0.65, 0.7, param_text, transform=plt.gca().transAxes, fontsize=12,
             verticalalignment='top', bbox=dict(boxstyle='round,pad=0.5', fc='wheat', alpha=0.5))
    
    # Format the plot
    plt.title('SEIAR Model with Piecewise Beta Fit to German COVID-19 Data', fontsize=16)
    plt.xlabel('Days Since 2020-03-01', fontsize=12)
    plt.ylabel('Infectious Population', fontsize=12)
    plt.legend(fontsize=11)
    plt.grid(True, which='both', linestyle='--', linewidth=0.5)
    plt.ylim(bottom=0)
    plt.xlim(left=0, right=len(t_data))
    
    # Save the figure
    plt.savefig('seiar_fit_germany_piecewise.png', dpi=300)
    print("\nFit result plot saved as 'seiar_fit_germany_piecewise.png'")
    
    # Display the plot
    plt.show()

except RuntimeError as e:
    print(f"Error: Could not complete the fit. Error message: {e}")
    print("This might be due to poor initial guesses or bounds. Try adjusting `initial_guesses` and `bounds`.")

except Exception as e:
    print(f"An unknown error occurred: {e}")
\end{lstlisting}

\subsection{Deterministic Four-Stage Model Script (script\_piecewise.py)}
\begin{lstlisting}[language=Python, caption={Python script for fitting a four-stage piecewise deterministic SEIAR model, aligning with the Bayesian model's structure.}]
import pandas as pd
import numpy as np
from scipy.integrate import odeint
from scipy.optimize import curve_fit
import matplotlib.pyplot as plt
import matplotlib.dates as mdates

def seiar_model(y, t, beta, N, f, sigma, gamma_i, gamma_a):
    """
    Standard SEIAR model ODE function for use by the internal solver.
    """
    S, E, I, A, R = y
    beta_a = 0.58 * beta  # Asymptomatic transmission rate is a fixed relative proportion
    
    force_of_infection = (beta * I + beta_a * A) * S / N
    
    dSdt = -force_of_infection
    dEdt = force_of_infection - sigma * E
    dIdt = (1 - f) * sigma * E - gamma_i * I
    dAdt = f * sigma * E - gamma_a * A
    dRdt = gamma_i * I + gamma_a * A
    
    return [dSdt, dEdt, dIdt, dAdt, dRdt]

def piecewise_seiar_model(t_data, beta1, beta2, beta3, beta4):
    """
    Piecewise SEIAR model for use with curve_fit.
    This function takes an array of time points and four beta parameters,
    and returns the predicted (I+A) curve for the entire period.
    """
    # Define breakpoints (consistent with the Bayesian model)
    breakpoints = [22, 71, 184]
    betas = [beta1, beta2, beta3, beta4]
    
    # Full time span
    t_full = np.arange(len(t_data))
    
    # Initialize the overall solution
    solution = np.array([]).reshape(0, 5)
    
    # Initial conditions
    # Get an estimate for initial E0 from the data (a small non-zero value)
    # We could also make E0 a fitting parameter, but for simplicity, we use a reasonable estimate here.
    initial_exposed = 500 
    current_y0 = [N - initial_exposed, initial_exposed, 0, 0, 0]
    
    # Solve the ODE segment by segment
    last_t = 0
    for i, bp in enumerate(breakpoints):
        # Define the time array for the current segment
        t_segment = np.arange(last_t, bp)
        
        # Solve the current segment
        sol_segment = odeint(
            seiar_model,
            y0=current_y0,
            t=t_segment,
            args=(betas[i], N, F, SIGMA, GAMMA_I, GAMMA_A)
        )
        
        # Concatenate the solution
        solution = np.vstack([solution, sol_segment])
        
        # Update initial conditions and start time for the next segment
        current_y0 = sol_segment[-1]
        last_t = bp

    # Solve the final segment
    t_segment_final = np.arange(last_t, len(t_full))
    sol_segment_final = odeint(
        seiar_model,
        y0=current_y0,
        t=t_segment_final,
        args=(betas[-1], N, F, SIGMA, GAMMA_I, GAMMA_A)
    )
    
    # Concatenate the final segment's solution
    solution = np.vstack([solution, sol_segment_final])

    # Return the model's predicted total infected population (I+A)
    return solution[:, 2] + solution[:, 3]


# --- 1. Load and prepare data ---
df_data = pd.read_csv('germany_covid_processed.csv', parse_dates=['date'])

# Define model parameters (based on literature)
F = 0.17
SIGMA = 1/4.5
GAMMA_I = 1/19.5
GAMMA_A = 1/17.0
N = df_data['population'].iloc[0]

# Prepare data for fitting (using the first 270 days)
fit_duration = 270
data_to_fit = df_data.head(fit_duration)
y_data = data_to_fit['infectious_pool'].values
x_data = np.arange(len(y_data))
dates = data_to_fit['date']

# --- 2. Perform piecewise deterministic fit ---
print("Performing piecewise deterministic fit...")

# Provide initial guesses and bounds for the four beta parameters
initial_guesses = [0.2, 0.1, 0.15, 0.1]
bounds = (0, [1.0, 1.0, 1.0, 1.0]) # Beta values are typically between 0 and 1

# Call curve_fit
popt, pcov = curve_fit(
    f=piecewise_seiar_model,
    xdata=x_data,
    ydata=y_data,
    p0=initial_guesses,
    bounds=bounds,
    method='trf' # 'trf' is suitable for bounded problems
)

# Extract optimal parameters
beta_opt = popt
print("\nFit complete! Optimal parameters are:")
for i, b in enumerate(beta_opt):
    print(f"  - beta_{i+1}: {b:.4f}")

# --- 3. Generate final fitted curve and plot ---
print("\nGenerating final fit plot...")
# Generate the final model curve using the optimal parameters
model_fit_curve = piecewise_seiar_model(x_data, *beta_opt)

# Plotting
plt.style.use('seaborn-v0_8-whitegrid')
fig, ax = plt.subplots(figsize=(14, 8))

# Plot real data
ax.plot(dates, y_data, 'o', markersize=4, color='dimgray', alpha=0.7, label='Observed Infectious Pool')

# Plot model fit curve
ax.plot(dates, model_fit_curve, color='dodgerblue', linewidth=2.5, label='Piecewise SEIAR Model Fit')

# Add vertical lines and labels for segments
breakpoints_days = [22, 71, 184]
for i, bp_day in enumerate(breakpoints_days):
    bp_date = dates.iloc[0] + pd.to_timedelta(bp_day, unit='D')
    ax.axvline(x=bp_date, linestyle='--', color='gray', linewidth=1.5)
    
# Add parameter labels
bp_dates = [dates.iloc[0]] + [dates.iloc[0] + pd.to_timedelta(d, unit='D') for d in breakpoints_days]
label_positions = [5, 35, 120, 200]
for i, pos in enumerate(label_positions):
     ax.text(dates.iloc[pos], max(y_data)*0.85, f'$\\beta_{i+1}={beta_opt[i]:.3f}$', 
             fontsize=12, backgroundcolor='white', color='blue')

# Format the plot
ax.set_title('Piecewise Deterministic SEIAR Model Fit to German COVID-19 Data', fontsize=16, pad=20)
ax.set_xlabel('Date', fontsize=12)
ax.set_ylabel('Estimated Infectious Population', fontsize=12)
ax.legend(fontsize=11)
ax.grid(True)

# Format date display
ax.xaxis.set_major_locator(mdates.MonthLocator(interval=2))
ax.xaxis.set_major_formatter(mdates.DateFormatter('%Y-%m'))
fig.autofmt_xdate()

# Save the figure
output_filename = 'seiar_fit_germany_piecewise.png'
plt.tight_layout()
plt.savefig(output_filename, dpi=300)

print(f"\nSuccess! Final fit plot saved as '{output_filename}'")
plt.show() 
\end{lstlisting} 